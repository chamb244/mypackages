\nonstopmode{}
\documentclass[a4paper]{book}
\usepackage[times,inconsolata,hyper]{Rd}
\usepackage{makeidx}
\usepackage[utf8]{inputenc} % @SET ENCODING@
% \usepackage{graphicx} % @USE GRAPHICX@
\makeindex{}
\begin{document}
\chapter*{}
\begin{center}
{\textbf{\huge Package `myf'}}
\par\bigskip{\large \today}
\end{center}
\ifthenelse{\boolean{Rd@use@hyper}}{\hypersetup{pdftitle = {myf: Random functions for better living.}}}{}
\begin{description}
\raggedright{}
\item[Type]\AsIs{Package}
\item[Title]\AsIs{Random functions for better living.}
\item[Version]\AsIs{1.0}
\item[Date]\AsIs{2022-12-08}
\item[Author]\AsIs{J. Chamberlin }\email{jordan.chamberlin@gmail.com}\AsIs{}
\item[Maintainer]\AsIs{Who to complain to }\email{jordan.chamberlin@gmail.com}\AsIs{}
\item[Description]\AsIs{This package holds random functions for recycling. }
\item[License]\AsIs{GPL3}
\end{description}
\Rdcontents{\R{} topics documented:}
\inputencoding{utf8}
\HeaderA{myf-package}{Random functions for better living.}{myf.Rdash.package}
\aliasA{myf}{myf-package}{myf}
%
\begin{Description}\relax
This package holds random functions for recycling. 
\end{Description}
%
\begin{Details}\relax

No description yet. 
Please be patient. 
The DESCRIPTION file:
This package was not yet installed at build time.\\{}

Index:  This package was not yet installed at build time.\\{}
\textasciitilde{}\textasciitilde{} An overview of how to use the package, including the most important functions \textasciitilde{}\textasciitilde{}
\end{Details}
%
\begin{Author}\relax
J. Chamberlin <jordan.chamberlin@gmail.com>

Maintainer: Who to complain to <jordan.chamberlin@gmail.com>
\end{Author}
\inputencoding{utf8}
\HeaderA{area\_circle}{Area of a circle}{area.Rul.circle}
\keyword{workshop}{area\_circle}
\keyword{self-improvement}{area\_circle}
\keyword{spatial}{area\_circle}
%
\begin{Description}\relax
Compute the area of a circle by providing the radius.
\end{Description}
%
\begin{Usage}
\begin{verbatim}
area_circle(radius)
\end{verbatim}
\end{Usage}
%
\begin{Arguments}
\begin{ldescription}
\item[\code{radius}] numeric. Cannot be negative.
\end{ldescription}
\end{Arguments}
%
\begin{Details}\relax
Nothing more to say...
\end{Details}
%
\begin{Value}
numeric
\end{Value}
%
\begin{Note}\relax
Apples and oranges....
\end{Note}
%
\begin{Author}\relax
j.chamberlin
\end{Author}
%
\begin{References}\relax
Pythagoras, 980
\end{References}
%
\begin{SeeAlso}\relax
\code{\LinkA{length}{length}}, \textasciitilde{}\textasciitilde{}\textasciitilde{}
\end{SeeAlso}
%
\begin{Examples}
\begin{ExampleCode}
 
area_circle(5)


## The function is currently defined as
function (radius) 
{
    if (!is.numeric(radius)) {
        warning("radius must be numeric")
        return(NA)
    }
    if (radius < 0) {
        warning("radius must be a positive value")
        return(NA)
    }
    area <- pi * radius * radius
    return(area)
  }
\end{ExampleCode}
\end{Examples}
\printindex{}
\end{document}
